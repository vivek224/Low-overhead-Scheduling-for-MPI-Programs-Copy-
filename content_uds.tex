\begin{frame}{OpenMP Loops Redux}
OpenMP provides a loop worksharing construct that specifies: 
\begin{itemize}
\item How the logical iteration space of a loop is cut into chunks
\item How the resulting chunks are assigned to the work threads of the parallel region.
\end{itemize}
  
%\begin{verbatim}
% # pragma omp for \[clause\[ \[,\] clause\] ... \]
%      for (int i=0; i<100; i++){}
%\end{verbatim}

\begin{itemize}
\item Loop needs to be in canonical form, that is, adhere to certain properties
\item The clause can be one or more of the following: \texttt{private(...)}, \texttt{firstprivate(...)}, \texttt{lastprivate(...)}, \texttt{linear(...)}, \texttt{reduction(...)}, \textbf{schedule(...)}, collapse(...), ordered[...], nowait, allocate(...) 
\item We focus on the clause \textbf{schedule()}
\end{itemize}

\end{frame}

\begin{frame}{Need Novel Loop Scheduling Techniques in OpenMP}

\begin{enumerate}
\item Supercomputer architectures and applications are changing.
\begin{itemize}
\small \item \small Large number of cores per node.
\item \small Speed variability across cores.
\item \small Complex dynamic behavior in applications themselves.
\end{itemize}
\item So, we need new methods of distributing an application’s parallelized loop's iterations to cores.
\item Such methods need to: 
\begin{itemize}
\small \item \small Ensure data locality, reduce synchronization overhead and maintain load balance~%\cite{worksteal99, dynwork2}.
\item \small Be aware of inter-node parallelism handled by libraries such as MPICH3~%\cite{mpich2}. 
\item \small Be suitable for the needs of a particular loops and machine characteristics.
\item \small Adapt during an application's execution.
\end{itemize}
\item Some customer demand for the SSG-DRD EMEA HPC team.
\end{enumerate}
\end{frame}




\begin{frame}{Utility of Novel Strategies Shown} 
\begin{itemize}
\item Utility of novel strategies is demonstrated in published work by V. Kale et al 1,2 and others.
\item For example, mixed static-dynamic scheduling strategy with an adjustable static fraction.
\item To limit the overhead of dynamic scheduling, while handling imbalances, such as those due to noise.
\end{itemize}
\end{frame}

\begin{frame}{Intel-specific Loop Schedules}
Intel’s (and LLVM's) OpenMP runtime offer additional scheduling types:
\begin{itemize}
\small \item \small E.g., static stealing
\item \small Are accessed through \texttt{schedule(runtime)} and \texttt{OMP\_SCHEDULE} environment variable
\item \small Cumbersome to use; very complex to extend (need to modify the RTL code and recompile the RTL code)
\item \small Not portable across OpenMP implementations
\item \small Proposed user-defined scheduling (UDS) will provide a portable and flexible way of extending OpenMP’s loop scheduling types.
\end{itemize}

\end{frame} 


\begin{frame}{ Specification of a User-defined Scheduling Scheme} 

\begin{enumerate}
\item We aim to specify a user-defined scheduling scheme within the OpenMP specification1 .
\item The scheme should accommodate an arbitrary user-defined scheduler.
\item These are the elements required to define a scheduler.
\begin{itemize}
\item Scheduler-specific data structures.
\item History record: adapt the loop schedule based on previous loop invocations and/or user-specified carry parameters. 
\item Specification of scheduling behavior of threads.
\end{itemize}
\end{enumerate} 
\end{frame}



\begin{frame}{ Proposal for a User-defined Schedule in OpenMP} 
This example gives a glimpse at how a user-defined schedule might look like:


\begin{figure}[ht!]
\lstinputlisting{./listings/omp_uds_example.c}
\end{figure}

\begin{itemize} 
\item The \texttt{declare schedule} directive is used to connect a schedule with a set of functions to initialize the schedule and hand out the next chunk of iterations.
\item The syntax of the \texttt{schedule} clause is extended to also accept an identifier of 
\item Instead of calling into the RTL for loop scheduling, the compiler will invoke the functions of the UDS 
\item Visibility and namespaces of these identifiers will be borrowed from UDRs
\end{itemize}  
\end{frame} 
%\begin{frame}{ An Implementation of the Dynamic Schedule with UDS} 
%\end{frame} 
%\begin{frame}{An Implementation of the Static Schedule with UDS}
%\end{frame} 

\begin{frame}{An Implementation of the Static/Dynamic Schedule with UDS}

\begin{figure}
    \centering 
    \lstinputlisting{listings/openmpudsdyndatstruct.c}
    \caption{Caption}
    \label{fig:my_label}
\end{figure}
\end{frame} 

\begin{frame}{Issues to Consider}
\begin{enumerate} 
\item Modifiers to clause: non-monotonic 
\begin{itemize}
\item Problem: How do we handle loop having indices that are non-monotonic?
\item Code example: 
\item Solution: For the proposal, we restrict users to to use monotonic loops for now
\end{itemize}
\item How do schedules guarantee correct execution when a global variables are used?
\begin{itemize}
\item Problem: 
\item Code example: 
\item Solution we propose: 
\end{itemize}

\item Compatibility with clause concurrent: 
\begin{itemize}
\item Problem: See e-mails from Xinmin et al on this, including definition of concurrent in the classical sense.
\item Code example showing problem:
\item Proposed Solution: can enforce that concurrent not be used with user-defined schedule. 
\end{itemize}
\end{enumerate}

\end{frame}


\begin{frame}{Implementation Specifics in LLVM} 

\begin{itemize}
\item Modify in LLVM OpenMP runtime. 
\begin{itemize}
\small \item \small Kmp\_sched.c
\end{itemize}


\item Modify LLVM's OpenMP compiler. 
\begin{itemize}
\small \item \small Would like help from people to do this to get a reference implementation done in a timely manner. 
\end{itemize} 

\end{itemize}

\end{frame}


\begin{frame}{Current Status of the Proposal}{}
\begin{itemize}
\item Vivek Kale (external) presented the proposal at OpenMPCon and IWOMP
\item There is a scientific paper showing the merits of UDS in general (using a prototype implementation based on macros)
\item A report out of Vivek and myself has been scheduled for the FTF in Austin in ww05
\item We are working on the syntax details plus a prototype implementation in the LLVM RTL (no compiler work envisioned for now)
\end{itemize}
\end{frame}



\begin{frame}[label=udsconclusions]{Conclusions} 
\begin{enumerate}
\item One can schedule an OpenMP loop using the clause schedule provided by OpenMP. 
\item The schedules available are defined by implementation.
\item Need for experimentation with sophisticated loop scheduling strategies.
\item OpenMP community should discuss how to allow flexible specification of such strategies in a user’s code and how to design a user-level scheduler library so that it can be portably used with any conforming OpenMP implementation.
\item Supporting user-defined schedulers in this way will facilitate rapid development of scheduling strategies.
\item We hope that the experts will discuss these ideas at this F2F.
\end{enumerate}
\end{frame}
